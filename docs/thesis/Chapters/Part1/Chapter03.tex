%************************************************
%************************************************
%************************************************
\chapter{Applications}\label{ch:applications}
%************************************************
%************************************************
%************************************************

%************************************************
%************************************************
\section{Scientific Applications}
%************************************************
%************************************************

%************************************************
\subsection{Neutron captures of astrophysical interest}
%************************************************

The main purpose of the \ac{i-TED} detector system is its application to nucleo-synthesis studies.

As previously discussed in \ref{ch:background}, the imaging capabilities of the detector allow for the selection of events based of their spacial origin. This results in an improvement of the peak-to-background ratio, allowing more precise measurements of the neutron capture cross-sections.

The cross-section of neutron captures are of interest for nuclear astrophysics because the abundance of elements in a start, and by extension its evolution, depend on the cross-section of different nuclei that form from it.

There are many cases where more precise measurements are needed in order to explain discrepancies between current theoretical models and the measured relative abundances of elements in the universe.

%************************************************
%************************************************
\section{Industrial Applications}
%************************************************
%************************************************

%************************************************
\subsection{Dose monitoring in hadron therapy}
%************************************************

Currently, the standard imaging techniques for radiotherapy are based on PET imaging which requires the presence of two detectors in front of each other with the source/target in the middle and which depends on the two 511 keV $\gamma$-rays from the electron-positron annihilation. Due to the need of two detectors facing each other the field of view is very limited requiring extensive detectors or arrays of detectors.

The use of Compton imaging enables the use of a single detector with a large field of view and the use of any $\gamma$-rays which can scatter off of the scatterer and be absorbed by the absorber. This greatly extends the efficiency of the detector since more events can be used both spatially and energy-wise. This also allows the use of prompt $\gamma$-rays instead of relying on the $\beta$-decay of short-lived (between bunches) or longer-lived (dedicated PET scanner) isotopes.

Nevertheless, both techniques have its place since, apart from the benefits of Compton aforementioned, PET also provides benefits since it is an easier imaging technique, provides better image resolution (depends mostly on timing resolution), and is more well established in the medical field.

Furthermore, since Compton cameras are $\gamma$-ray detectors, if two are positioned in front of each other with the source/target in the middle, both Compton and PET techniques can be used. This fusion of techniques is studied in \cite{}.

This conjunction of techniques can also be used for validation and further development of Compton imaging algorithms.

%************************************************
\subsection{Nuclear waste analysis}
%************************************************

One increasingly common application for imaging detectors is the analysis of nuclear waste.

In many cases the nuclear waste, specially that which is highly radioactive, doesn't have the same level of radioactivity throughout the whole material. More commonly, waste classified as having a higher level of radiation is composed of a smaller very radioactive portion and a bigger less radioactive portion. 

The use of imaging capabilities allows the identification of more problematic parts of the waste that can be separated, treated and stored according to the respective procedures without treating the whole waste as a monolith. This has significant improvements in terms of storage space and cost of treatment.

Imaging capabilities are also of great interest to analyze the activation in nuclear facilities, both industrial and research, that can help improve safety precautions and possible structural damage.