%************************************************
%************************************************
%************************************************
\chapter{Introduction}\label{ch:background}
%************************************************
%************************************************
%************************************************

The multi \ac{i-TED} detector system was developed within the \ac{HYMNS}-\ac{ERC} project by the Gamma-ray and Neutron Spectroscopy group of \ac{IFIC}.\\

The main objective of the \ac{HYMNS}-\ac{ERC} project is the study of the s-process through neutron capture experiments at \ac{CERN}'s \ac{nToF}, described in \ref{sec:nucleosynthesis}.\\

The system was developed to address a common problem of neutron capture experiments, the relatively high background in the range of energies of interest. Incidentally, many of the background events result of neutrons from the beam or $\gamma$ from the target interacting with the surrounding materials, exciting them, and the subsequent emission of $\gamma$ radiation during their de-excitation.\\

To that end, an array of Compton cameras was developed in order to select events based on their spacial origin using the imaging capabilities of the detector. The principle behind this type of detector is described in \ref{sec:interaction}. The detector itself is described in \ref{ch:detector} and characterized in \ref{ch:multi-ited}.\\

%************************************************
%************************************************
\section{Nucleosynthesis}\label{sec:nucleosynthesis}
%************************************************
%************************************************

The question of the relative abundances of the different elements is one of the central topics of Nuclear Astrophysics and the one that birthed the field through the concept of nucleosynthesis in the first half of the XX century.

Nucleosynthesis is the process that leads to the creation of atomic nuclei through different nuclear reactions, in different conditions.

The present thesis concern's itself mainly with stellar nucleosynthesis, meaning the creation of nuclei through the nuclear processes that fuel stellar evolution. This process is closely related to the relative abundance of elements lighter than iron.

The main nuclear processes that occur in stellar nucleosynthesis are:
\begin{itemize}
    \item s-process
    \item r-process
    \item p-process
    \item rp-process
\end{itemize}

The \ac{i-TED} detector was developed to study the neutron cross-sections of s-process reactions.

%************************************************
\subsection{s-process}
%************************************************


%************************************************
\subsection{Neutron time of flight experiments}
%************************************************

A common method of studying the neutron capture cross-section over a large range of energies is by using a neutron time of flight experiment.

The \ac{HYMNS}-\ac{ERC} project has connections with the following neutron time of flight facilities:
\begin{itemize}
    \item \ac{nToF}, \ac{CERN}
    \item \ac{HISPANoS}, \ac{CNA}
    \item \ac{ILL}
\end{itemize}

<EXPLAIN>

\begin{figure}[h!]
    \centering
    \includegraphics[width=5cm]{}%
    \caption{Neutron time of flight experiments.}%
    \label{fig:ntof}%
\end{figure}

%************************************************
%************************************************
\section{Interaction of radiation with matter}\label{sec:interaction}
%************************************************
%************************************************

The current version of the \ac{i-TED} detector system is a $\gamma$ imager. As such the interactions that happen between the electromagnetic radiation and the scintillator crystal, resulting in the detection, are described in this section.

%************************************************
\subsection{Compton scattering}
%************************************************

%************************************************
\subsection{Scintillators}
%************************************************

%************************************************
%************************************************
\section{Machine Learning}\label{sec:ml}
%************************************************
%************************************************

%************************************************
\subsection{Concepts}
%************************************************

\subsubsection{Underfitting and overfitting}

\subsubsection{Bias, trade-offs and validation}

\subsubsection{Supervised and unsupervised learning}

%************************************************
\subsection{Models}
%************************************************

\subsubsection{Regression}

\subsubsection{Clustering}

\paragraph*{k-Means}

\paragraph*{Hierarchical}

\paragraph*{Density-based}

\subsubsection{Neural Networks}

\paragraph*{Convolutional Neural Networks}

The term \ac{CNN} refers to a specific type of \ac{NN} that uses the convolution operation in one of its layers, \cite{Goodfellow2016}. This type of \ac{NN} is specialized for grid-like data (such as pixels in an image), according to \cite{Goodfellow2016, VALUEVA2020232}.

\paragraph*{Graph Neural Networks}

The term \ac{GNN} refers to a specific type of \ac{NN} that is represented, not as simple grid data structures, but rather in graph-structured data, \cite{GNN2022}. According to \cite{GNN2022}, the technique has shown promising results in the field of computer vision, including processing images, video and cross-media.