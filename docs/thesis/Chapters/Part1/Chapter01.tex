%************************************************
%************************************************
%************************************************
\chapter{Introduction}\label{ch:background}
%************************************************
%************************************************
%************************************************

The multi \ac{i-TED} detector system was developed within the \ac{HYMNS}-\ac{ERC} project by the Gamma-ray and Neutron Spectroscopy group of \ac{IFIC}.\\

The main objective of the \ac{HYMNS}-\ac{ERC} project is the study of the s-process through neutron capture experiments at \ac{CERN}'s \ac{nToF}, described in \ref{sec:nucleosynthesis}.\\

The system was developed to address a common problem of neutron capture experiments, the relatively high background in the range of energies of interest. Incidentally, many of the background events result of neutrons from the beam or $\gamma$ from the target interacting with the surrounding materials, exciting them, and the subsequent emission of $\gamma$ radiation during their de-excitation.\\

To that end, an array of Compton cameras was developed in order to select events based on their spacial origin using the imaging capabilities of the detector. The principle behind this type of detector is described in \ref{sec:interaction}. The detector itself is described in \ref{ch:detector} and characterized in \ref{ch:multi-ited}.\\

%************************************************
%************************************************
\section{Nuclear Astrophysics}\label{sec:astrophysics}
%************************************************
%************************************************

%************************************************
\subsection{Nucleosynthesis}\label{sec:nucleosynthesis}
%************************************************

The question of the relative abundances of the different elements is one of the central topics of Nuclear Astrophysics and the one that birthed the field through the concept of nucleosynthesis in the first half of the XX century.

Nucleosynthesis is the process that leads to the creation of atomic nuclei through different nuclear reactions, in different conditions.

The present thesis concern's itself mainly with stellar nucleosynthesis, meaning the creation of nuclei through the nuclear processes that fuel stellar evolution. This process is closely related to the relative abundance of elements lighter than iron.

The main nuclear capture processes that occur in stellar nucleosynthesis are:

\begin{itemize}
    \item s-process
    \item r-process
    \item p-process
    \item rp-process
\end{itemize}

The \ac{i-TED} detector was developed to study the neutron cross-sections of s-process reactions.

%************************************************
\subsection{Neutron capture processes}
%************************************************

In contrast with the isotopes up to $^{56}$Fe, which were mainly created by fusion, isotopes with higher mass number were formed by neutron capture reactions such as the s- and r-processes, as well as by proton capture precesses such as p- and rp-processes.

The neutron capture processes are classified based on the relative timescale of the neutron capture and neutron decay by $\beta^-$.

When the neutron capture happens slower than the $\beta^-$ decay, the process is slow (s-process). If otherwise the neutron capture is faster than the $\beta^-$ decay, the process is rapid (r-process).

\begin{itemize}
    \item s-process: $a$
    \item r-process: $a$
\end{itemize}

If s- or r-process takes place will depend on the neutron capture cross-section, on the neutron flux inside the star, and the half-life of the $\beta^-$ decay branch.

%************************************************
\subsection{Challenges}
%************************************************

\paragraph*{Energy range}


\paragraph*{Cross-section}


\paragraph*{Background}

%************************************************
\subsection{Neutron time of flight experiments}
%************************************************

A common method of studying the neutron capture cross-section over a large range of energies is by using a neutron time of flight experiment.

The \ac{HYMNS}-\ac{ERC} project has connections with the following neutron time of flight facilities:
\begin{itemize}
    \item \ac{nToF}, \ac{CERN}
    \item \ac{HISPANoS}, \ac{CNA}
    \item \ac{ILL}
\end{itemize}

<EXPLAIN>

\begin{figure}[h!]
    \centering
    \includegraphics[width=5cm]{}%
    \caption{Neutron time of flight experiments.}%
    \label{fig:ntof}%
\end{figure}

%************************************************
%************************************************
\section{Interaction of radiation with matter}\label{sec:interaction}
%************************************************
%************************************************

The current version of the \ac{i-TED} detector system is a $\gamma$ imager. As such the interactions that happen between the electromagnetic radiation and the scintillator crystal, resulting in the detection, are described in this section.

%************************************************
\subsection{Compton scattering}
%************************************************

There are three mechanisms of interaction between $\gamma$-rays, and electromagnetic radiation in general, and matter. The probability of interaction by each method depends mainly on the energy of the interacting radiation and the atomic number of the absorber it interacts with. The main interaction method is described in \ref{fig:interaction_mechanism}.

\begin{figure}[h!]
    \centering
    \includegraphics[width=5cm]{gfx/Interaction of Radiation with Matter.png}%
    \caption{Most probable interaction mechanism of electromagnetic radiation with matter as a function of energy and atomic number Z. From \cite[p.~7]{ParticlePhysicsReference2}.}%
    \label{fig:interaction_mechanism}%
\end{figure}

As the detector is an array of Compton cameras, the main interaction of interest is Compton scattering.

%************************************************
\subsection{Scintillators}
%************************************************

%************************************************
%************************************************
\section{Machine Learning}\label{sec:ml}
%************************************************
%************************************************

%************************************************
\subsection{Concepts}
%************************************************

\subsubsection{Underfitting and overfitting}

\subsubsection{Bias, trade-offs and validation}

\subsubsection{Supervised and unsupervised learning}

%************************************************
\subsection{Models}
%************************************************

\subsubsection{Regression}

\subsubsection{Clustering}

\paragraph*{k-Means}

\paragraph*{Hierarchical}

\paragraph*{Density-based}

\subsubsection{Neural Networks}

\paragraph*{Convolutional Neural Networks}

The term \ac{CNN} refers to a specific type of \ac{NN} that uses the convolution operation in one of its layers, \cite{Goodfellow2016}. This type of \ac{NN} is specialized for grid-like data (such as pixels in an image), according to \cite{Goodfellow2016, VALUEVA2020232}.

\paragraph*{Graph Neural Networks}

The term \ac{GNN} refers to a specific type of \ac{NN} that is represented, not as simple grid data structures, but rather in graph-structured data, \cite{GNN2022}. According to \cite{GNN2022}, the technique has shown promising results in the field of computer vision, including processing images, video and cross-media.