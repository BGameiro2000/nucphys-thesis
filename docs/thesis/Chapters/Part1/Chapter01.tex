%************************************************
%************************************************
%************************************************
\chapter{Applications}\label{ch:applications}
%************************************************
%************************************************
%************************************************

%************************************************
%************************************************
\section{Scientific Applications}
%************************************************
%************************************************

%************************************************
\subsection{Neutron captures of astrophysical interest}
%************************************************



The main purpose of the i-TED detector is its application to 

%************************************************
%************************************************
\section{Industrial Applications}
%************************************************
%************************************************

%************************************************
\subsection{Dose monitoring in hadron therapy}
%************************************************

\begin{itemize}
    \item SYNERGY4HT
    \begin{itemize}
        \item Using imaging capabilities for studying the range in tissue
        \item To be used for cancer treatment 
        \item The energy deposition is mostly at the very end of the path (Bragg's peak)
        \item Generates prompt g that are then detected by i-TED by compton imaging
        \item Also allows the use of PET by selecting only 511keV events in one detector in opposite deteectors
    \end{itemize}
    \item Compton vs PET
    \begin{itemize}
        \item PET provides better image resolution
        \item PET is an easier technique
        \item PET is performed either between bunches (for short lived [ms] isotopes) or in a dedicated PET scanner (for longer lived [min] isotopes)
        \item Compton provides much more counts (uses different energies, PET only uses the 511keV from anhilation)
        \item Compton uses prompt g-rays occuring in the vicinity of the Bragg peak
        \item PET occurs in the anhilation of B-radiation emitted by an isotope, being correlated to the Bragg peak but not easily identifyable from data without simulations or just providing relative measurements of the range
        \item Providing both Compton and PET, i-TED can be used to improve MC simulations thar relate the PET range spectrum to the Bragg peak
    \end{itemize}
\end{itemize}

%************************************************
\subsection{Nuclear waste analysis}
%************************************************

\begin{itemize}
    \item Victor's PhD
    \begin{itemize}
        \item Using i-TED for this application
        \item Data collected on the reaction 80Se(n,g)
        \item Data being discussed with theoreticians
    \end{itemize}
    \item Application to ESARDA 2023 - Nuclear Safeguards and Non Proliferation
    \begin{itemize}
        \item https://esarda.jrc.ec.europa.eu/events/21st-esarda-course-2023-04-24_en
        \item Applied on Feb 6th - Waiting
        \item Course online on 24 - 28 April, 2023
        \item 3 ECTS
    \end{itemize}
\end{itemize}