%************************************************
%************************************************
%************************************************
\chapter{i-TED}\label{ch:detector}
%************************************************
%************************************************
%************************************************

The i-TED setup, consisting of 4 i-TED detectors, is a total energy detector which profits from its imaging capabilities in order to suppress background events that can be classified as not coming from the target, a novel approach in neutron capture measurements.

%************************************************
%************************************************
\section{Total Energy Detectors}
%************************************************
%************************************************

%************************************************
\subsection{Objective}
%************************************************

%************************************************
\subsection{Conditions}
%************************************************

%************************************************
\subsection{Linearization}
%************************************************

%************************************************
%************************************************
\section{Imaging Detectors}
%************************************************
%************************************************

%************************************************
\subsection{Compton Cameras}
%************************************************

\subsubsection{Compton scattering}

\subsubsection{Working principle}

%************************************************
\subsection{PET}
%************************************************

\subsubsection{Annihilation}

\subsubsection{Working principle}

%************************************************
\subsection{Comparison of methods}
%************************************************

%************************************************
%************************************************
\section{The detector}
%************************************************
%************************************************

%************************************************
\subsection{Description}
%************************************************

\subsubsection{Detecting material}

\subsubsection{Neutron absorber}

\subsubsection{Silicon photomultiplier}

\subsubsection{Acquisition system}

\subsubsection{Geometry}

\subsubsection{Distance - }


%************************************************
\subsection{Distinguishing features}
%************************************************

\subsubsection{Size \& efficiency}

\paragraph{paragraph}

\subsubsection{Applicability in neutron capture}

\paragraph{Detecting material}\label{par:LaCl3}

\paragraph{Neutron absorber}

Although, as described in \autoref{par:LaCl3}, the scintillator is chosen for its lower neutron sensitivity, in order to further reduce the overall neutron sensitivity of the system $^6$LiH, a neutron moderator, is placed covering the front plane of each scatterer.

%************************************************
\subsection{Characterization}
%************************************************

Upon the start of this thesis project, the i-TED detectors had not been used for several months, since the completion of its campaign at the n\_ToF facility of CERN in 2022. It was also placed on the experimental hall of IFIC-UV, whereas its last characterization took place in the underground facilities of CERN.

With both the time and difference in conditions of the last characterization in mind, a new one was performed. 

\subsubsection{Preliminary characterization}

Taking into account the complexity of the i-TED system, which consists of 4 different i-TEDs and a total of 20 different crystals, a preliminary and simpler characterization is performed to ensure that no part of the system has degraded to the point of needing to be replaced.

This preliminary characterization consists in calculating the energy resolution for the 662keV peak of Cs-137, a common benchmark that had previously been used for the current setup, \cite{}.

Furthermore, as the previous calibration was used in this step, the position of the cesium peak was used to assess how the calibration had deteriorated.

For this purpose the Cs-137 source was placed between the absorber and scatterer planes of each i-TED. Each measurement was performed for 300 seconds, for the 4 i-TEDs, for 2 different configurations of the PETSyS acquisition system (8.8.8 and 8.8.11), corresponding to different thresholds. Measurements with a third configuration, 8.8.5, were performed for 60 seconds, due to the amount of data obtained by lowering the threshold.

Analyzing the data, the mean energy resolution at 662keV are presented in the table below:


This is ---- describe.

The crystal iTEDB-4 had the most noticeable change, table x, and (is it going to be changed?)

