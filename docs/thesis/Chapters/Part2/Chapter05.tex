%************************************************
%************************************************
%************************************************
\chapter{i-TED-E}\label{ch:e}
%************************************************
%************************************************
%************************************************

%************************************************
%************************************************
\section{Motivation}\label{sec:motivation-e}
%************************************************
%************************************************

With the conclusion of the multi \ac{i-TED} upgrade, characterization and its setup in a configuration that allows its use for both neutron capture experiments and dose monitoring in hadron therapy, the need for an individual \ac{i-TED} detector module arose.

The individual \ac{i-TED}, hereby \ac{i-TED}-E, is to be used in situations that don't warrant the need for multiple Compton cameras, such as for testing or nuclear waste analysis.

This detector was built using spare parts, including components that were new or taken from the multi \ac{i-TED} during the upgrade.

As this detector is not expected to perform in conditions with high flux of neutron or \gamma, it has two differences to the other modules. First, it doesn't include the --- shield in from of the scatterer. Second, it has a full metal casing in front of the absorber which facilitates its mounting and a consistent application of pressure over its crystals.

%************************************************
%************************************************
\section{Characterization}\label{sec:characterization-e}
%************************************************
%************************************************

%************************************************
\subsection{Energy resolution}
%************************************************

\subsubsection{Singles resolution}

\subsubsection{Coincidences resolution}

%************************************************
\subsection{Image reconstruction}
%************************************************

%************************************************
\subsection{Counting rate}
%************************************************

%************************************************
\subsection{Alpha activity}
%************************************************

%************************************************
\subsection{Timing resolution}
%************************************************