%************************************************
%************************************************
%************************************************
\chapter{i-TED-E}\label{ch:e}
%************************************************
%************************************************
%************************************************

%************************************************
%************************************************
\section{Motivation}\label{sec:motivation-e}
%************************************************
%************************************************

With the conclusion of the multi \ac{i-TED} upgrade, characterization and its setup in a configuration that allows its use for both neutron capture experiments and dose monitoring in hadron therapy, the need for an individual \ac{i-TED} detector module arose.

The individual \ac{i-TED}, hereby \ac{i-TED}-E, is to be used in situations that don't warrant the need for multiple Compton cameras, such as for testing or nuclear waste analysis.

This detector was built using spare parts, including components that were new or taken from the multi \ac{i-TED} during the upgrade.

As this detector is not expected to perform in conditions with high flux of neutron or $\gamma$, it has two differences to the other modules. First, it doesn't include the Li-6 shield in front of the scatterer. Second, it has a full metal casing in front of the absorber which facilitates its mounting and a consistent application of pressure over its crystals.

%************************************************
%************************************************
\section{Characterization}\label{sec:characterization-e}
%************************************************
%************************************************

%************************************************
\subsection{Energy resolution}
%************************************************

As the detector has crystals and \ac{SiPM} that were discarded from the multi \ac{i-TED}, including those that presented poor energy resolution and "clouds", the performance of \ac{i-TED}-E is expected to be considerably degraded in comparison with the results presented in \ref{sec:characterization}.

\subsubsection{Singles resolution}

A series of 25 runs of 15s was performed with a Cs-137 source. The files were processed with 100ns and 200ns coincidence windows between planes and the optimized 88C threshold based of the 887 configuration.

For the whole detector, \ac{i-TED}-E presented an overall resolution at the 662keV Cs-137 peak of 8.46 and 8.42 for 100ns and 200ns, respectively.

However, this overall value hides the considerable discrepancies between crystals presented in \ref{}.

[TABLE WITH MEAN VALUES]

This indicates that by changing the components related to absorber 3 and 4 or outright discarding the results from those crystals (which results in a decrease to half of the events) would improve the statistics considerably. By discarding events from crystals 3 and 4 the resolution would improve to [] and [].

\subsubsection{Coincidences resolution}

%************************************************
\subsection{Image reconstruction}
%************************************************

%************************************************
\subsection{Counting rate}
%************************************************

%************************************************
\subsection{Alpha activity}
%************************************************

From the series of 25 runs and both coincidence windows configurations, the results on \ref{} for the alpha activity rate were obtained. The alpha activity rate was calculated as described in \ref{}.

[TABLE]

The results presented show, once again, the performance variance between crystals.

%************************************************
\subsection{Timing resolution}
%************************************************